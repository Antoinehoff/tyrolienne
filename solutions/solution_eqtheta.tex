\par\vspace{2mm}
{\textit{Théorème du moment cinétique en $G$}}\\
Au centre de masse $G$ on obtient
\begin{equation}
     \frac{\mathrm d \bm{L}_G}{\mathrm d t} = \sum \bm M_G.\label{eq:thmMC}
\end{equation}
Le moment cinétique est donné part l'équation
\begin{equation}
    \bm L_G = \sum_i \overrightarrow{GM}_i \times m_i \bm v_i^*= I_G \dot\theta \ez.
\end{equation}
La somme des moments s'écrit
\begin{equation}
    \sum \bm M_G = \overrightarrow{GA}\times \bm N + \overrightarrow{GG}\times m\bm g
\end{equation}
où le deuxième terme tombe car la gravité n'induit pas de moment par rapport au centre de masse ($\overrightarrow{GG}=0$).
Le moment induit par la force de soutien s'écrit
\begin{equation}
  \overrightarrow{GA}\times \bm N = -l(\sin(\theta)\ex - l\cos(\theta) \ey)\times N\ey= -lN\sin(\theta) \ez.
\end{equation}
L'équation Eq. \eqref{eq:thmMC} s'écrit donc selon $\ez$
\begin{equation}
    I_G \ddot\theta + lN\sin(\theta) = 0.\label{eq:thmMC_ez}
\end{equation}