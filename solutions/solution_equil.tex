\par\vspace{2mm}
En imposant les conditions d'équilibre ($\theta = \theta_{eq} = \textrm{const}$, $\dot \theta = 0 $ et $\ddot \theta = 0$) dans Eq. \eqref{eq:Eqdumvt} on obtient
\begin{equation}
     \omega_0^2 \sin(\theta_{eq}) \cos(\alpha)  = 0. \label{eq:thetaeq}
\end{equation}
Une solution est $\omega_0^2 =0$ ce qui équivaut à annuler la gravité o\`u à un pendule infiniment long (peu intéressant ici).
Une autre est donnée par $\theta_{eq}=n\pi$, $n \in \mathbb N$. 
On obtient ainsi que la barre est en équilibre lorsqu'elle forme un angle droit avec le câble.
Finalement, $\alpha=\pi/2+n\pi$ satisfait aussi la condition d'équilibre.
Dans ce cas, tout angle $\theta \in \mathbb R$ est un équilibre car la barre est en chute libre.
On résume les différents équilibres pour $\omega_0^2\neq 0$, i.e.
\begin{equation}
    \theta_{eq}= 
\begin{cases}
    n\pi,& \text{si } \alpha\neq\pi/2 + n'\pi\\
    \theta\in\mathbb R, & \text{sinon}.
\end{cases}
\label{eq:thetaeqs}
\end{equation}

On a résolu ici le cas général. 
L'équilibre $\theta=0$ correspond au cas $n=0$.

\note{On ne peut pas trouver la position d'équilibre en imposant $\mathrm d E_{pot}/\mathrm d\theta =0$ ici car l'équilibre n'est pas défini par rapport à un référentiel d'inertie. La somme des forces n'est pas nulle.}