\par\vspace{2mm}
On multiplie l'Eq. \eqref{eq:newton_ey} par $l\sin\theta$ et on utilise Eq. \eqref{eq:thmMC_ez} pour obtenir une équation différentielle pour $\theta(t)$ qui dépend uniquement des paramètres du problème
\begin{equation}
    \left[\sin^2(\theta) +\eta\right]\ddot\theta +\frac{1}{2}\sin(2\theta)\dot\theta^2 + \omega_0^2\sin(\theta) \cos(\alpha) = 0 \label{eq:Eqdumvt}
\end{equation}
avec $\omega_0^2=g/l$ la fréquence propre d'un pendule de longueur $l$ et $\eta=I_G/ml^2$ un nombre sans dimension décrivant l'effet de la géométrie de la barre par rapport à un pendule de longueur $l$. 
