\par\vspace{2mm}
Les forces en présence sont la gravité $m\bm g$ et le soutien du câble $\bm N$.
Pour projeter ces forces on pose un premier référentiel d'inertie lié à la terre $\mathcal R$ avec l'origine $O$ posée sur le câble. 
En plus des coordonnées cartésiennes $(x,y,z)$ définit un deuxième référentiel $\mathcal R'$ lié au point d'attache $A$ et on se munit d'un système de coordonnées cylindrique auxiliaire $(r,\theta,z)$ dont l'origine est en $A$.
Le vecteur unitaire $\er$ est orienté selon $\overrightarrow{AG}$.
L'angle $\theta$ mesure l'angle entre la tyrolienne et l'axe perpendiculaire au câble.
Ces informations sont rassemblées sur la Figure \ref{fig:schema_details}.
\par 
On a les équivalences
\begin{align}
    \ex &= \sin\theta \er + \cos\theta \et,    \label{eq:ex_rt}\\
    \ey &= -\cos\theta \er + \sin\theta \et
    \label{eq:ey_rt}
\end{align}
et
\begin{align}
    \bm g &= g\sin\alpha \ex - g\cos\alpha \ey\nonumber\\
        &= g[\sin\alpha\sin\theta + \cos\alpha\cos\theta]\er + g[\sin\alpha\cos\theta - \cos\alpha\sin\theta]\et\nonumber\\
        &= g\cos(\theta-\alpha)\er - g\sin(\theta-\alpha)\et
\end{align}
Ainsi, la projection des forces dans le repère cylindrique est donnée par
\begin{align}
    m\bm g &= mg\cos(\theta-\alpha) \er - mg\sin(\theta-\alpha) \et,\\
    \bm N  &= N \bm e_y.
\end{align}

Le théorème du centre de masse stipule que l'on peut appliquer la seconde loi de Newton en considérant chaque force appliquée sur la barre au centre de masse G.
On obtient alors 
\begin{align}
    m \bm a_G &= \sum \bm F \nonumber\\
    \Leftrightarrow m (\bm a'_G + \bm a_A) &= m\bm g + \bm N \label{eq:newtonCM}
\end{align}
o\`u $\bm a_G$ est l'accélération du centre de masse selon le référentiel d'inertie lié à $O$ et $\bm a'_G$ selon le référentiel accéléré lié à $A$. $\bm a_A$ est l'accélération relative du référentiel lié en $A$ par rapport au référentiel d'inertie.
En exprimant ces différentes accélérations dans le système de coordonnée cylindrique on obtient
\begin{align}
    \bm a'_G &= (\ddot r - r\dot\theta^2)\er + (2\dot r \dot \theta + r \ddot \theta) \et  + \ddot z \ez \\
    \bm a_A &=\ddot x_A\ex= \ddot x_A \sin\theta \er + \ddot x_A \cos\theta \et
\end{align}
Étant donné que la tyrolienne est un solide indéformable, le centre de masse ne peut pas se déplacer selon $\er$. Ceci impose $r=l=\textrm{const}$ et donc $\dot r = \ddot r =0$.
On développe à présent l'Eq. \eqref{eq:newtonCM} selon les directions $\er$ et $\et$, i.e.
\begin{align}
    -ml\dot\theta^2 + m\ddot x_A\sin\theta &= -N\cos\theta + mg\cos(\theta-\alpha) \label{eq:NewtonCM_er}\\
    ml\ddot\theta + m\ddot x_A\cos\theta &= N\sin\theta - mg\sin(\theta-\alpha)\label{eq:NewtonCM_et}.
\end{align}
L'équation du mouvement selon $\ez$ est triviale.
Il est aussi possible de retrouver la seconde loi de Newton selon $\ex$ et $\ey$ en les combinant selon les équivalences Eqs. \eqref{eq:ex_rt} et \eqref{eq:ey_rt}, i.e.
\begin{align}
ml\ddot\theta\cos\theta - ml\dot\theta^2 \sin\theta  + m\ddot x_A &= m g \sin\alpha\label{eq:newton_ex}\\
ml\ddot\theta\sin\theta +ml \dot\theta^2\cos\theta &= N - mg\cos\alpha \label{eq:newton_ey}
\end{align}

 \begin{figure}
     \centering
     \includegraphics[width=0.5\linewidth]{figures/tyrolienne_schemas_details_cable.pdf}
     \caption{Schéma détaillé du problème}
     \label{fig:schema_details}
 \end{figure}